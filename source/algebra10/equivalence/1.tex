%%%%%%%%%%%%%%%%%%%%%%%%%%
% Эквивалентные переходы %
%%%%%%%%%%%%%%%%%%%%%%%%%%

% Тихонов Юлий, 2011-2012

\emph{Далее всюду $c > 0$.}
\vspace{-2.5ex}

\subsection*{Иррациональные уравнения}

\begin{tabbing}
\hspace{20em}\=\\[-4ex]
\(
    \sqrt{f} = g
\Leftrightarrow
    \begin{cases}
        f = g^2
    ;\\
        g \geq 0
    .\end{cases}
\)
\>
\problem
$\sqrt{3 + x} = 3 - x$.
\\[2ex]%%%%%%%%%%
\begin{minipage}{20em}
\(
    \sqrt{f} = \sqrt{g} + c
\Leftrightarrow
    f = g + c^2 + 2 c \sqrt{g}
\).
\\[0.5ex]
\text{Что будет при $c \leq 0$?}
\end{minipage}
\>
\problem
$\sqrt{x + 5} = \sqrt{x - 3} + 2$.
\\[2ex]%%%%%%%%%%
\(
    \sqrt{f} + \sqrt{g} = c
\Leftrightarrow
    \begin{cases}
        2 \sqrt{fg} = c^2 - f - g
    ;\\
        f \geq 0
    ;\quad
        g \geq 0
    .\end{cases}
\)
\>
\problem
$\sqrt{x + 1} + \sqrt{x - 4} = 5$.
\\[2ex]%%%%%%%%%%
\(
    \sqrt{f} + \sqrt{g} = \sqrt{h}
\Leftrightarrow
    \begin{cases}
        f + g + 2 \sqrt{fg} = h
    ;\\
        f \geq 0
    ;\quad
        g \geq 0
    .\end{cases}
\)
\>
\problem
$\sqrt{3 x + 1} + \sqrt{x + 4} = \sqrt{9 - x}$.
\end{tabbing}

\claim{Замечание о монотонности}
В уравнении вида $F(x) = c$ в случае строго монотонной функции $F$ можно угадать
корень, и он будет единственный.

\example
\\[0.5ex]
\textbf{(1)}
\(
    \sqrt{2 x - 3} + \sqrt{4 x + 1} = 4
\).
Левая часть возрастает, $x = 2$~--- корень.
\\[1ex]
\textbf{(2)}
\(
    \sqrt{x} - \sqrt{x - 5} = 1
\Leftrightarrow
    \dfrac{5}{\sqrt{x} + \sqrt{x - 5}} = 1
\).
Левая часть убывает, $x = 9$~--- корень.


\subsection*{Иррациональные неравенства}

\begin{tabbing}
\hspace{20em}\=\\[-4ex]
\(
    \sqrt{f} < g
\Leftrightarrow
    \begin{cases}
        f < g^2
    ;\\
        f \geq 0
    ;\quad
        g \geq 0
    .\end{cases}
\)
\>
\problem
$\sqrt{4 - x} < x + 2$.
\\[2ex]%%%%%%%%%%
\(
    \sqrt{f} > g
\Leftrightarrow
    \left[\begin{aligned}
        &f > g^2
    ;\\
        &\begin{cases}
            g < 0
        ;\\
            f \geq 0
        .\end{cases}
    \end{aligned}\right.
\)
\>
\problem
$\sqrt{x + 5} > 7 - x$.
\\[2ex]%%%%%%%%%%
\(
    \sqrt{f} - \sqrt{g} > c
\Leftrightarrow
    f > g + c^2 + 2 c \sqrt{g}
\).
\>
\problem
$\sqrt{6 + x} - \sqrt{4 - x} > 2$.
\\[2ex]%%%%%%%%%%
\(
    \sqrt{f} - \sqrt{g} < c
\Leftrightarrow
    \begin{cases}
        f < g + c^2 + 2 c \sqrt{g}
    ;\\
        f \geq 0
    .\end{cases}
\)
\>
\problem
$\sqrt{x + 5} - \sqrt{x - 3} < 2$.
\\[2ex]%%%%%%%%%%
\(
    \sqrt{f} + \sqrt{g} \vee c
\Leftrightarrow
    \begin{cases}
        f + g + 2 \sqrt{f g} \vee c^2
    ;\\
        f \geq 0
    ;\quad
        g \geq 0
    .\end{cases}
\)
\>
\problem
$\sqrt{2 x + 3} + \sqrt{3 x + 3} > 1$.
\\[2ex]%%%%%%%%%%
\(
    \sqrt{f} + \sqrt{g} > \sqrt{h}
\Leftrightarrow
    \begin{cases}
        f + g + 2 \sqrt{f g} > h
    ;\\
        f \geq 0
    ;\quad
        g \geq 0
    ;\\
        h \geq 0
    .\end{cases}
\)
\>
\problem
$\sqrt{4 x - 23} + \sqrt{x + 3} > \sqrt{x + 10}$.
\\[2ex]%%%%%%%%%%
\(
    \sqrt{f} + \sqrt{g} < \sqrt{h}
\Leftrightarrow
    \begin{cases}
        f + g + 2 \sqrt{f g} < h
    ;\\
        f \geq 0
    ;\quad
        g \geq 0
    .\end{cases}
\)
\>
\problem
$\sqrt{x - 1} + \sqrt{2 x - 3} < \sqrt{x + 2}$.
\end{tabbing}


\subsection*{Логарифмы}

$\veebar$~--- это одно из $>$, $<$, $\geq$, $\leq$;
\quad
$\vee$~--- это одно из $>$, $<$.

\begin{tabbing}
\hspace{20em}\=\\[-4ex]
\(
    \log_a f = \log_a g
\Leftrightarrow
    \begin{cases}
        f = g > 0
    ;\\
        a \in (0; 1) \cup (1; +\infty)
    .\end{cases}
\)
\>
\problem
$\log_5 (x - 1) = \log_5 \dfrac{x}{1 + x}$.
\\[2ex]%%%%%%%%%%
\(
    \log_a f > \log_a g
\Leftrightarrow
    \left[\begin{aligned}
        &\begin{cases}
            f > g > 0
        ;\\
            a > 1
        ;\end{cases}
     \\
        &\begin{cases}
            0 < f < g
        ;\\
            f > 0
        ;\\
            0 < a < 1
        .\end{cases}
    \end{aligned}\right.
\)
\>
\problem
$\log_7 \dfrac{x + 3}{21} > \log_7 \dfrac{2}{3 x - 6}$.
\\[2ex]%%%%%%%%%%
\(
    \log_a f + \log_a g \veebar 0
\Leftrightarrow
    \begin{cases}
        \log_a (f g) \veebar 0
    ;\\
        f > 0
    .\end{cases}
\)
\>
\problem
$\lg (x + \frac{3}{2}) + \lg x > 0$.
\\[2ex]%%%%%%%%%%
\(
    \log_a f - \log_a g \veebar 0
\Leftrightarrow
    \begin{cases}
        \log_a \dfrac{f}{g} \veebar 0
    ;\\
        f > 0
    .\end{cases}
\)
\>
\problem
$\log_3 \dfrac{x + 1}{3} - \log_3 \dfrac{x}{2 - x} > 0$.
\\[2ex]%%%%%%%%%%
\(
    \log_g f > c
\Leftrightarrow
    \left[\begin{aligned}
        &\begin{cases}
            g > 1
        ;\\
            f > g^c
        ;\end{cases}
     \\
        &\begin{cases}
            0 < g < 1
        ;\\
            0 < f < g^c
        .\end{cases}
    \end{aligned}\right.
\)
\>
\problem
$\log_{3 x - 1} 2 x > 1$.
\\[2ex]%%%%%%%%%%
\(
    \log_g f < c
\Leftrightarrow
    \left[\begin{aligned}
        &\begin{cases}
            g > 1
        ;\\
            0 < f < g^c
        ;\end{cases}
     \\
        &\begin{cases}
            0 < g < 1
        ;\\
            f > g^c
        .\end{cases}
    \end{aligned}\right.
\)
\>
\problem
$\log_x \dfrac{4 x + 1}{6(x - 1)} < 1$.
\\[2ex]%%%%%%%%%%
%\(
%    \log_g f \vee c
%\Leftrightarrow
%    \begin{cases}
%        f > 0
%    ;\\
%        g > 0
%    ;\\
%        (g - 1) (f - g^c) \vee 0
%    .\end{cases}
%\)
%\>
%%
%\\[2ex]%%%%%%%%%%
\begin{minipage}{20em}
\(
    \log_g f \veebar c
\Leftrightarrow
    \begin{cases}
        f > 0
    ;\\
        g > 0
    ;\\
        \cfrac{f - g^c}{g - 1} \veebar 0
    .\end{cases}
\)
\end{minipage}
\>
\problem
$\log_{9 x^2} (6 + 2 x - x^2) \leq 1$.
\end{tabbing}

\claim{Пример (о монотонности)}
\\
\textbf{(1)}
$\log_3 (x + 3) + \log_3 (x + 1) = 1$.
Левая часть возрастает, $x = 0$~--- корень.
\\
\textbf{(2)}
$2 \lg x > \log_2 (12 - x) + 1$.
Левая часть возрастает, правая~--- убывает, $x = 10$~--- корень.


\subsection*{Бонус}

\emph{Решить с помощью эквивалентных преобразований.}

\begin{problems}

\item
\(
    \dfrac{
        26 - 3 x + \sqrt{x^2 - 2 x - 24}
    }{
        x - 10
    }
<
    -1
\).
\qquad
\problem
$\sqrt{x + 1} - \sqrt{2 x - 5} - \sqrt{x - 2} = 0$.

\item
% не имеет отношения к иррациональным неравенствам.
\(
    \sqrt{x + 3 - 4 \sqrt{x - 1}}
    +
    \sqrt{x + 8 - 6 \sqrt{x - 1}}
=
    1
\).

\item
$\log_{1/3} (3 x + 4) > \log_{1/9} (x^2 + 2)$
\qquad
\problem
$\log_x (9 x + 1) < \log_x |10 x - 1|$.

\item
$\log_{2 |x|} (x^2 - 13 x + 36) \leq \log_{2 |x|} 6$.
\qquad
\problem
\(
    \dfrac{
        \log_{1/2}
            \frac{12 x^2 - 60 x + 77}{4}
    }{
        x^2 - 5 x + 6
    }
<
    0
\).

\item
\(
    \dfrac{
        \lg (3 x^2 - 3 x + 7) - \lg (6 + x - x^2)
    }{
        (10 x - 7) (10 x - 3)
    }
\geq
    0
\).
\qquad
\problem
\(
    \log_{x + 5/2}
        \left(
            \dfrac{x - 5}{2 x - 3}
        \right)^2
>
    0
\).

\end{problems}

