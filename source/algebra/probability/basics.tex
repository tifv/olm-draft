%%%%%%%%%%%%%%%%%%%%%%%%%%%%%%
% Вероятности: комбинаторика %
%%%%%%%%%%%%%%%%%%%%%%%%%%%%%%

% originally by Часовских

\begin{problems}

\item
В последовательности чисел $1$, $2$, \ldots, $n$ отмечено число $k$.
Найдите вероятность того, что из двух наудачу выбранных чисел этой
последовательности одно меньше $k$, а другое больше.
(Числа выбираются наудачу независимо~--- в частности, могут совпадать.)

\item
Бросается $n$ игральных костей.
Найдите вероятность того, что выпадет
$n_1$ единиц, $n_2$ двоек, \ldots, $n_6$ шестерок
($n_1 + n_2 + \ldots + n_6 = n$).

\item
Последовательность чисел $1$, $2$, \ldots, $4N$ разбивается наудачу на две
равные группы.
Найдите вероятность того, что
\\
\sbp в каждой группе будет поровну четных и нечетных чисел;
\\
\sbp все числа, кратные $N$, окажутся в первой группе;
\\
\sbp числа, кратные $N$, поделятся поровну между группами.

\item
Сколько раз нужно бросить игральную кость, чтобы появление $6$ очков хотя бы в
одном из бросков имело вероятность большую $0.8$?

\item
Двое по очереди вынимают шары (без возвращения) из ящика, содержащего
$M$ белых и $N - M$ черных шаров (отличающихся только цветом).
Какова вероятность того, что первый белый шар будет вынут начинающим?
Рассмотреть случай $N = 6$, $M = 2$.

\item
Докажите, что если
$A \cap B \cap C \subset D$,
то
$P(A) + P(B) + P(C) \leq 2 + P(D)$.

\item
Каждую секунду с вероятностью $p$ независимо от других моментов времени по
дороге проезжает автомобиль.
Для перехода дороги пешеходу необходимо $3\,\text{c.}$
Какова вероятность того, что подошедший к дороге пешеход будет ожидать
возможность перехода
\quad
\sbp $4\,\text{с.}$
\quad
\sbp $5\,\text{с.}$

\item
% трэш
Найдите вероятность того, что наудачу взятое целое число окажется простым,
предполагая, что вероятность того, что оно кратно $n$, равна $\frac{1}{n}$.

\item
Рабочий обслуживает три станка, на которых обрабатываются однотипные детали.
Вероятность брака для первого станка равна $0.02$, для второго~--- $0.03$,
для третьего~--- $0.04$.
Обработанные детали складываются в один ящик.
Производительность первого станка в три раза больше, чем второго, а
третьего~--- в два раза меньше, чем второго.
Определите вероятность того, что взятая наудачу деталь будет бракованной.

\end{problems}
