%%%%%%%%%%%%%%%%%%%%%%%%%%%%%%%%%
% Комплексные числа в геометрии %
%%%%%%%%%%%%%%%%%%%%%%%%%%%%%%%%%

% Whoever, 2011

\subsection*{Преобразования}

\emph{Арифметическими действиями}
\emph{(операциями)}
будем называть сложение, умножение, вычитание, деление и комплексное
сопряжение.
Под <<выразить>> будем иметь в виду
<<выразить с помощью пяти арифметических действий>>.

\begin{problems}

\item
\emph{Поворот.}
Пусть точка $z'$ получена из точки $z$ поворотом на
\\
\sbp
$90^\circ$
\quad
\sbp
$60^\circ$
\quad
\sbp
угол $\alpha$
\quad
вокруг нуля. Выразите $z'$ через $z$ и $\alpha$.
\\
\sbp вокруг точки $w$.  Выразите $z'$ через $z$, $w$ и $\alpha$.

\item
\emph{Растяжение.}
Пусть точка $z'$ получена из точки $z$ растяжением с коэффициентом $k$ и с
центром
\\
\sbp в нуле. Выразите $z'$ через $z$ и $k$.
\\
\sbp в точке $w$. Выразите $z'$ через $z$, $w$ и $k$.

\item
\emph{Cимметрия.}
Пусть точки $z'$ и $z$ симметричны относительно
\\
\sbp
мнимой оси.
Выразите $z'$ через $z$.
\\
\sbp
прямой, проходящей через точки $w_1$ и $w_2$.
Выразите $z'$ через $z$, $w_1$ и $w_2$.

\item
\emph{Инверсия.}
Пусть точка $z'$~--- образ точки $z$ при инверсии относительно
\\
\sbp
единичной окружности с центром в нуле.
Выразите $z'$ через $z$.
\\
\sbp
окружности с радиусом $r$ и с центром в $w$.
Выразите $z'$ через $z$, $w$ и $r$.

\item
Докажите, что композиция инверсий с разными центрами~--- не инверсия и не
подобие.

\end{problems}

\definition
\emph{Дробно-линейным преобразованием}
называется преобразование плоскости, задаваемое формулой
\[
    z \mapsto \frac{a z + b}{c z + d}
\quad \text{ или } \quad
    z \mapsto \frac{a \bar z + b}{c \bar z + d}
,\qquad
    a, b, c, d \in \CC
,\]
где величина $a d - b c$, называемая \emph{определителем} преобразования, не
равна нулю.

\begin{problems}

\item
Разложите произвольное дробно-линейное преобразование в композицию инверсий и
подобий.

\item
Композиция двух дробно-линейных преобразований~--- тоже д.\,л.\,п.
Докажите это.
Найдите коэффициенты результирующего д.\,л.\,п.

\end{problems}


\subsection*{Основные ГМТ}

\begin{problems}

\item
Докажите, что точка $z$ лежит на одной прямой с точками $z_1$ и $z_2$, если и
только если
\[
    \frac{z - z_1}{z_2 - z_1}
\in
    \RR
.\]

\item
Докажите, что уравнение серединного перпендикуляра к отрезку с концами $z_1$ и
$z_2$ имеет вид
\(
    z
    (\bar z_1 - \bar  z_2)
    +
    \bar z
    (z_1 - z_2)
    -
    |z_1|^2
    +
    |z_2|^2
=
    0
\).

\item
Докажите, что четыре точки $z_1$, $z_2$, $z_3$, $z_4$ лежат на одной окружности
или прямой тогда и только тогда, когда
\[
    \frac{z_1 - z_3}{z_2 - z_3}
    \colon
    \frac{z_1 - z_4}{z_2 - z_4}
\in
    \RR
.\]

\end{problems}


\subsection*{Геометрические задачи}

\emph{Засчитываются только верные алгебраические решения.}

\begin{problems}

\item
Треугольники
$\triangle A B C$,
$\triangle A B_A C_A$,
$\triangle A_C B_C C$,
$\triangle A_B B C_B$
одинаково ориентированы и подобны при указанном порядке обхода вершин.
Докажите, что середины отрезков
$A_B A_C$, $C_A C_B$, $B_C B_A$ образуют треугольник, подобный $\triangle ABC$.

\item
На плоскости даны два квадрата $A_1 B_1 A_2 C_1$ и $A_2 B_2 A_3 C_2$ с общей
вершиной $A_2$.
Пусть $O_1$ и $O_2$~--- центры квадратов,$B$~--- середина отрезка $B_1 B_2$,
$C$~--- середина отрезка $C_1 C_2$.
Докажите, что $O_1 B O_2 C$~--- квадрат.

\item
Точки $A$ и $B$ движутся равномерно и с \emph{одинаковыми} угловыми скоростями
по окружностям $\Omega_1$ и $\Omega_2$ соответственно (против часовой стрелки).
Докажите, что вершина $C$ правильного треугольника $ABC$ также движется
равномерно по некоторой окружности.

\item
На единичной окружности отмечены $n$ точек, являющиеся вершинами правильного
$n$-угольника.
Одна из них соединена хордами с остальными.
Найти произведение длин этих хорд.

\item
Каждую сторону $n$-угольника в процессе обхода против часовой стрелки
продолжили на ее длину.
Оказалось, что концы построенных отрезков служат вершинами правильного
$n$-угольника.
Докажите, что исходный $n$--угольник тоже правильный.

\itemx{*}
Дан равносторонний треугольник $ABC$.
Найдите геометрическое место точек $X$, т.ч.
$\angle XAB + \angle XBC + \angle XCA = 90^{\circ}$
(углы ориентированные).

\itemx{*}
\emph{Теорема Клиффорда.}
Пусть дано $n$ прямых общего положения на плоскости
(никакие две не параллельны, никакие три не пересекаются в одной точке).
Для четного $n$ существует \emph{центральная точка} $n$ прямых, а для нечетного
$n$ существует \emph{центральная окружность} ($n > 1$).
Центральные объекты определятся так: для $n = 2$ центральная точка~--- точка
пересечения прямых; для $n + 1$ центральный объект (точка или окружность)~---
это объект, инцидентный всем центральным объектам всех поднаборов из $n$
прямых.
Докажите существование центральных объектов.
\\
Отношение \emph{инцидентности}~--- отношение принадлежности.
Точка и окружность инцидентны, если точка лежит на окружности.

\end{problems}

