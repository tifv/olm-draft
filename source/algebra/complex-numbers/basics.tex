%%%%%%%%%%%%%%%%%%%%%
% Комплексные числа %
%%%%%%%%%%%%%%%%%%%%%

% Whoever, 2011

\begin{gather*}
    z = a + b \ii
\qquad
    \bar z = a - b \ii
\qquad
    \Re z = a
\qquad
    \Im z = b
\qquad
    |z|^2 = a^2 + b^2
\\
    \Re z = \frac{z + \bar z}{2}
\qquad
    \Im z = \frac{z - \bar z}{2 \ii}
\qquad
    |z|^2 = z \cdot \bar z
\end{gather*}

\begin{problems}

\item
Представьте в тригонометрической форме числа
\\
\sbp $1 + \ii$;
\quad
\sbp $2 + \sqrt{3} + \ii$;
\quad
\sbp $1 + \cos \phi + \ii \sin \phi$;
\\[1ex]
\sbp $\sin(\pi / 6) + \ii \sin(\pi / 6)$;
\quad
\sbp $\dfrac{\cos \phi + \ii \sin \phi}{\cos \varphi - \ii \sin \varphi}$.

\item
Вычислите
\\
\sbp $(1 + \ii)^n$;
\quad
\sbp $(1 + \ii \sqrt{3})^n$;
\quad
\sbp $\biggl(\dfrac{1 + \ii \sqrt{3}}{1 - \ii}\biggr)^{20}$;
\quad
\sbp $\biggl(1 - \dfrac{\sqrt{3} - \ii}{2}\biggr)^{20}$;
\\[1ex]
\sbp $(1 + \cos \phi + \ii \sin \phi)^n$;
\quad
\sbp
\(
    \biggl(
        \dfrac{\cos \phi + \ii \sin \phi}{\cos \psi + \ii \sin \psi}
    \biggr)^n
\).

\item
Решите уравнения:
\quad
\sbp $z^2 + z + 1 = 0$;
\quad
\sbp $z^2 - (3 + 2 \ii) z + 6 \ii = 0$;
\\
\sbp $z^{-1} = \bar z$;
\quad
\sbp $z^4 = \bar z^4$.
\quad
\sbp $z^2 + |z| = 0$.

\item
Докажите, что из
$|x| = |y| = |z| = 1$
следует
\\
\sbp $|x + y + z| = |x y + x z + y z|$;
\quad
\sbp $\dfrac{x y z}{(x + y) (x + z) (y + z)} \in \RR$.

\item
Докажите, что
$\dfrac{z - 1}{z + 1} \in \ii \RR$
эквивалентно
$|z| = 1$, $z \not= -1$.

\item
Упростите сумму
$S_n = 1 + 2 \cos x + 2^2 \cos 2 x + \ldots + 2^n \cos(n x)$.

\item
Дано комплексное число
\(
    z
=
    \dfrac{1 + \lambda \ii}{1 - \lambda \ii}
\),
где $\lambda \in \RR$.
Докажите, что $z^n + \bar z^n = 2 \cos(2 n \arctg \lambda)$.

\item
При каких $n$ число
\(
    z
=
    \left(
        \dfrac{3 + \ii}{2 - \ii}
    \right)^n
\)
действительное?

\item
Найти ГМТ $z$, таких что $|z - 1| = 2 |z + 1|$.

\item
Найдите все целые $m$, такие что $(1 + \ii)^m = (1 - \ii)^m$.

\item
Вычислите модуль и аргумент комплексного числа
$\dfrac{z + 1}{z - 1}$,
где
$z = \cos \phi + \ii \sin \phi$.

\item
В координатной плоскости $O x y$ дана последовательность точек
$T_j(x_j, y_j)_{j \geq 0}$:
\[
    x_0 = 1
,\quad
    y_0 = 0
,\quad
    x_{j + 1} = \sqrt{3} x_j - y_j
,\quad
    y_{j + 1} = x_j + \sqrt{3} y_j
.\]
В какой четверти находится точка $T_{2011}$?

\item
Комплексное число $z$ удовлетворяет соотношению
\(
    \left|
        \dfrac{z - \ii}{z - 2 \ii}
    \right|
=
    \dfrac{1}{2}
\). 
Каким может быть $|z|$?

\item
Решите в комплексных числах систему
$|z - 1| = 2$,
$|z + 2 - 4 \ii| = 3$.

\item
Решите в комплексных числах уравнение
$|z - 2| + |z - 5 + 4 \ii| = 5$.

\item
Решите
\quad
\sbp $z^n = 1$;
\quad
\sbp $z^{n - 1} + z^{n - 2} + \ldots + z + 1 = 0$;
\\
\sbp $z^n = w$, где $w \in \CC$~--- константа.

\item
Комплексные числа $a, b, c$ удовлетворяют соотношению
$a^2 + b^2 + c^2 = a b + b c + c a$.
Докажите, что $a = b = c$ либо $a, b, c$~--- вершины равностороннего
треугольника.
\\Верно ли обратное?

\item
В окружность радиуса $1$ вписан правильный $n$-угольник
$A_1 A_2 \ldots A_n$.
Чему равно произведение
$A_1 A_2 \cdot A_1 A_3 \cdot \ldots \cdot A_1 A_n$?

\end{problems}


\subsection*{Суммы}

\begin{problems}

\item
Вычислите суммы
\\[1ex]
\sbp
\(
    \dbinom{100}{0} - \dbinom{100}{2} + \dbinom{100}{4}
    - \ldots +
    \dbinom{100}{100}
\);
\\[1ex]
\sbp
\(
    \dbinom{99}{1} - \dbinom{99}{3} + \dbinom{99}{5}
    - \ldots +
    \dbinom{99}{99}
\);
\\[1ex]
\sbpx{*}
\(
    \dbinom{n}{0} + \dbinom{n}{3} + \dbinom{n}{6}
    + \ldots
\);
\qquad
\sbpx{*}
\(
    \dbinom{n}{1} + \dbinom{n}{4} + \dbinom{n}{7} + \ldots
\).

\end{problems}


\subsection*{Ещё задачи}

\begin{problems}

\item
Решите уравнения:
\\
\sbp $z^2 + \ov{z} = 0$;
\quad
\sbp $z^2 + |z|^2 = 0$;
\quad
\sbp $(z + \ii)^4 = (z - \ii)^4$;
\quad
\sbp $z^3 - \ov{z} = 0$;

\item
Разложите выражение $a^3 + b^3 + c^3 - 3 a b c$ на линейные множители.

\item
Решите в комплексных числах систему уравнений
$x^4 + 6 x^2 y^2 + y^4 = 5$,
$x^3 y + x y^3 = 1$.

\item
Комплексное число $a + b \ii$ является корнем уравнения $x^3 + p x + q = 0$,
причем $a, b, p, q \in \RR$, $b \neq 0$.
Докажите, что $a q > 0$.

\end{problems}

