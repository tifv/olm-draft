%%%%%%%%%%%%%%%%%%
% Четыре теоремы %
%%%%%%%%%%%%%%%%%%

\theoremof{Вильсона}
Если $p$~--- простое число, то
\[
    (p - 1)!
\equiv
    -1 \pmod p
.\]

\theoremof{Ферма}
Пусть $p$ простое, $a$ целое.
Тогда
\[
    a^p \equiv a \pmod p
.\]

\definition
\emph{Функция Эйлера}~--- это количество чисел от $0$ до $m-1$, взаимно простых с $m$:
\[
    \phi(m)
=
    \#
    \bigl\{
        t
    \bigm|
        0 \leq t < m
    ,\quad
        (t, m) = 1
    \bigr\}
\]

\theoremof{Эйлера}
Пусть $(a, m) = 1$.
Тогда
\[
    a^{\phi(m)} \equiv 1 \pmod m
.\]

\claim{Китайская теорема об остатках (КТО)}
Пусть дано $n$ попарно взаимно простых натуральных чисел
$m_{1}$,~$\ldots$,~$m_{n}$.
Пусть также даны целые числа $a_1$, $\ldots$, $a_n$.
Тогда система уравнений
\[
\left\{\begin{aligned}
    x &\equiv a_1 \pmod {m_1}
,\\[-5pt]
    &\;\;\vdots
\\[-5pt]
    x &\equiv a_n \pmod {m_n}
,\end{aligned}\right.
\]
эквивалентна уравнению вида
\[
    x \equiv a \pmod {M}
\]
для некоторого $a$, где $M = m_1 \cdot m_2 \cdot \ldots \cdot m_n$


\subsection*{Теорема Вильсона}

\begin{problems}

\item
Вычислите $(n - 1)! \pmod n$ в зависимости от $n$.

\item
Докажите, что простое число $p = 4 k + 1$ делит $((2 k)!)^2 + 1$.

\end{problems}


\subsection*{Теорема Ферма}

\begin{problems}

\item
Найдите остаток от деления
\\
\sbp $3^{100}$ на $101$;
\quad
\sbp $3^{102}$ на $101$;
\quad
\sbp $3^{99}$ на $101$;
\quad
\sbp $2^{90}$ на $101$.

\item
Пусть $n$~--- натуральное число, не кратное $17$.
Докажите, что либо $n^8 + 1$, либо $n^8 - 1$ делится на 17.

\item
Для каких $n$ число $n^{2011} - n^{4}$ делится на 11?

\item
Докажите, что при любом целом $a$
\\
\sbp $30 \mid a^5 - a$;
\quad
\sbp $510 \mid a^{17} - a$;
\quad
\sbp $66 \mid a^{11} - a$;
\\
\sbp
\(
    2 \cdot 3 \cdot 5 \cdot 7
    \cdot
    13 \cdot 19 \cdot 37 \cdot 73
\mid
    a^{73} - a
\);
\sbp $561 \mid a^{561} - a$.

\item
Сумма трех чисел $a$, $b$ и $c$ делится на 30.
Докажите, что $a^5 + b^5 + c^5$ тоже делится на $30$.

\item
Пусть $p$ и $q$~--- различные простые числа.
Докажите, что
\[
    p^q + q^p \equiv p + q \pmod{pq}
\]

\item
Докажите, что число $30^{239} + 239^{30}$ составное.

\item
Докажите, что для любого простого числа $p \neq 2, 5$ существует такое
натуральное число $m$, что
\(
    p
\mid
    \underbrace{11\ldots1}_{m}
\).

\item\label{problem:ord lemma}%
Пусть $p$ простое, и
$a^k \equiv 1 \pmod p$, причем $k$~--- наименьшее такое число, то есть для
$0 < r < k$ выполнено $a^r \not\equiv 1 \pmod p$.
Докажите, что $k \mid p - 1$.

\item
\sbp
Докажите, что если $p \mid x^2 + 1$, то либо $p = 2$, либо $p = 4 k + 1$.
\\
\emph{Подсказка: используйте \ref{problem:ord lemma}.}
\\
\sbp
Докажите, что простых чисел вида $4 k + 1$ бесконечно много.

\item
\sbp
Докажите, что если $p \mid x^2 + x + 1$, то либо $p = 3$, либо $p = 6 k + 1$.
\\
\sbp
Докажите, что простых чисел вида $6 k + 1$ бесконечно много.

\item
\sbp
Докажите, что если $p \mid x^4 + x^3 + x^2 + x + 1$, то либо $p = 5$, либо
$p = 5 k + 1$.
\\
\sbp
Докажите, что простых чисел вида $5 k + 1$ бесконечно много.

\end{problems}


\subsection*{Функция Эйлера}

\begin{problems}

\item
Вычислите $\phi(p)$, где $p$~--- простое.

\item
Вычислите $\phi(p^n)$, где $p$~--- простое, $n$~--- натуральное.

\item
Известно, что
$
    \phi(382997)
%=
%    \phi(449) \phi(853)
=
    381696
$.
Чему равно $\phi(382997^2)$?

\item
Выразите $\phi(a^n)$ через $\phi(a)$ и $a$.

\item
Решите уравнение $2 \phi(n) = \phi(2 n)$.

\item
Пусть $(m, n) = 1$.
Рассмотрим таблицу $m \times n$ ($m$ строчек, $n$ столбцов), в которую
{\it подряд} записаны числа $1, 2, \ldots, mn$.
\begin{figure}[!ht]
\begin{center}
\begin{tabular}{cccc}
$1$             &$2$                &$\ldots$     &$n$\\
$n + 1$         &$n + 2$            &$\ldots$     &$2 n$\\
$\vdots$        &$\vdots$           &$\ddots$     &$\vdots$\\
$n (m - 1) + 1$ &$n (m - 1) + 2$    &$\ldots$     &$nm$
\end{tabular}
\end{center}
\end{figure}
\\
\sbp Где расположены числа, взаимно простые с $n$?
\\
\sbp Докажите, что в каждом столбце есть {\it все остатки} по модулю $m$.
\\
\sbp Докажите, что $\phi(m n) = \phi(m) \phi(n)$.

\item
Поставьте знак нестрогого неравенства: $\phi(a b) \vee \phi(a) \phi(b)$.
Когда достигается равенство?

\item
Пусть $a = p_1^{\alpha_1} p_2^{\alpha_2} \ldots p_n^{\alpha_n}$.
Докажите, что
\[
    \phi(a)
    =
    a
    \left(
        1 - \frac{1}{p_1}
    \right)
    \left(
        1 - \frac{1}{p_2}
    \right)
    \ldots
    \left(
        1 - \frac{1}{p_n}
    \right)
.\]

\item
Докажите, что если $n > 2$, то $\phi(n)$ четно.

\item
Докажите, что $\phi(\phi(n)) \leq n/2$ при $n > 6$.

\item
Докажите, что для любого $n \in \NN$ справедливо
\[
    \sum_{d \mid n}
        \phi(d)
=
    n
\]
(сумма по положительным делителям числа $n$).

\end{problems}


\subsection*{Теорема Эйлера}

\begin{problems}

\item
Найдите остаток от деления
\\
\sbp $3^{84}$ на $100$;
\quad
\sbp $3^{78}$ на $100$;
\quad
\sbp $2^{90}$ на $100$.

\item
Докажите, что для любого нечетного числа $n$ существует такое натуральное число
$m$, что $n \mid 2^m - 1$.

\item
Докажите, что для любого положительного числа $n$ такого, что
$(n, 2) = (n, 5) = 1$,
существует такое натуральное число $m$, что
\(
    n
\mid
    \underbrace{11\ldots1}_{m}
\).

\item
Найдите все $m$, для которых равенство
\[
    a^{\phi(m) + 1}
\equiv
    a \pmod m
\]
выполнено для всех целых $a$.

\end{problems}


\subsection*{Китайская теорема об остатках}

\begin{problems}

\item
Докажите, что, какое бы ни было $N \in \NN$, найдутся $N$ подряд идущих
натуральных чисел
$a_1$, $a_2$, $\ldots$, $a_N$
таких, что
\[
    p_1^{2010} \mid a_1
,\quad
    p_2^{2010} \mid a_2
,\quad
\ldots
,\quad
    p_N^{2010} \mid a_N
,\]
где $p_1 = 2$, $p_2 = 3$, $p_3 = 5$, \ldots, $p_N$~--- первые $N$ простых
чисел.

\end{problems}

