%%%%%%%%%%%%%%%%%%%%%%
% Признаки делимости %
%%%%%%%%%%%%%%%%%%%%%%

\[
    \ov{a_1 a_2 \ldots a_n}
=
    a_1 10^{n - 1} + a_2 10^{n - 2} + \ldots + a_{n - 1} 10^1 + a_n
\]

\begin{problems}

\item
Докажите, что любое натуральное число дает такой же остаток при делении на
$10$, что и его последняя цифра в десятичной записи.
То же при делении на $2$, то же при делении на $5$.

\item
Докажите, что остаток от деления натурального числа на $4$ совпадает с
остатком от деления числа, составленного из двух его последних цифр в
десятичной записи. 

\item
Сформулируйте признаки делимости на $2^n$, $5^n$.

\item
Последняя цифра квадрата натурального числа равна $6$.
Докажите, что его предпоследняя цифра нечетна.

\item
Докажите, что степень двойки не может оканчиваться четырьмя одинаковыми
цифрами.

\item
Найдите $100$-значное число без нулевых цифр, которое делится на сумму своих
цифр.

\item
Докажите, что любое натуральное число дает такой же остаток при делении на $3$,
что и сумма его цифр.
То же при делении на $9$.

\item
У числа $2^{100}$ нашли сумму цифр, у результата снова нашли сумму цифр и
т.\,д.
В конце концов получилось однозначное число.  Найдите его.

\item
Докажите, что если записать в обратном порядке цифры любого натурального числа,
то разность исходного и нового числа будет делиться на $9$.

\item
К числу $15$ припишите слева и справа по одной цифре так, чтобы полученное
число делилось на $15$.

\item
Найдите наименьшее натуральное число, делящееся на $36$, в записи которого
встречаются все $10$ цифр.

\item
Может ли сумма цифр точного квадрата равняться $1970$?

\item
Докажите, что $\ov{a_1 a_2 \ldots a_n}$ дает такой же остаток при делении на
11, что и $a_n - a_{n - 1} + \ldots + (-1)^{n - 1} a_1$.

\item
В десятизначном числе все цифры встречаются по разу.
Может ли оно делиться на $11$?

\item
Докажите, что число $111\ldots11$
(всего $2n$ единиц)~--- составное.

\item
$A$~--- шестизначное число, в записи которого по одному разу встречаются цифры
$1$, $2$, $3$, $4$, $5$, $6$.
Докажите, что $A$ не делится на $11$.

\item
Найдите все натуральные числа, которые увеличиваются в $9$ раз, если между
цифрой единиц и цифрой десятков вставить $0$.

\item
К числу справа приписывают тройки.
Докажите, что когда-нибудь получится составное число.

\item
Внимательно посмотрев на последовательные тройки цифр в десятичной записи числа,
сформулируйте и докажите признаки делимости на $7$, $13$ и $37$.

\item
Докажите, что числа вида $\overline{AAA}$ делятся на $3$ и на $37$.

\item
Найдите все двузначные числа, равные сумме числа своих десятков и квадрата
числа единиц.

\item
Найдите сумму цифр всех чисел последовательности $1$, $2$, $\ldots$, $10000$.

\item
Некоторое число $k$ в десятичной системе записывается при помощи семерок,
шестерок и пятерок.
Причем количество пятерок и семерок равное.
Докажите, что число $k + 15$~--- составное.

\item
Дана полоска из $6$ клеток.
Петя и Вася по очереди записывают в клетки цифры.
Если полученное число делится на $91$, то выигрывает Вася, если нет, то Петя.
Всегда ли Вася сможет победить?

\item
$25! = 15511210043330985984\mathbf{?}00000$.
Определите, какую цифру заменили на $\mathbf{?}$.

\item
Если двухзначное число разделить на сумму его цифр, то получится в частном $6$ и
в остатке $2$.
Если же это число разделить на произведение его цифр, то получится в частном
$5$ и в остатке $2$.
Найдите это число.

\item
Является ли число $12345678926$ квадратом?

\item
Вася написал на доске пример на умножение двух двухзначных чисел, а затем
заменил в нем все цифры на буквы, причем одинаковые цифры~--- на одинаковые
буквы, а разные~--- на разные.
В итоге у него получилось $AB \cdot CD = EEFF$.
Докажите, что он где-то ошибся.

\item
Может ли число, записываемое при помощи $100$ нулей, $100$ единиц и $100$
двоек, быть точным квадратом?

\end{problems}

