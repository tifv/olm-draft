%%%%%%%%%%%%%
% Делимость %
%%%%%%%%%%%%%

\definition
Пусть $a$~--- целое число, $b$~--- натуральное.
Существует единственная пара целых чисел $q$ и $r$ таких, что
$a = b \cdot q + r$ и $0 \leq r < b$.
Числа $q$ и $r$ называются соответственно
\emph{(неполным) частным} и \emph{остатком} при делении $a$ на $b$.

\definition
Натуральное число $p > 1$, которое имеет ровно два натуральных делителя
($1$ и $p$), называется \emph{простым}.
Натуральные числа, отличные от $1$ и не являющиеся простыми, называются
\emph{составными}.
По соглашению, число $1$ не относят ни к простым, ни к составным числам.

\claim{Основная теорема арифметики}
Любое натуральное число представимо в виде произведения простых сомножителей.
Такое представление единственно с точностью до порядка сомножителей.

\definition
Два числа называются \emph{взаимно простыми}, если у них нет общих делителей,
отличных от 1.

\begin{problems}

\item
\sbp
Если некоторое число делится на два взаимно простых числа $n$ и $m$, то оно
делится и на их произведение $nm$.
\\
\sbp\emph{Основная лемма арифметики.}
Если число $p \cdot A$ делится на $q$, где $p$ и $q$ взаимно просты, то и $A$
делится на $q$.

\end{problems}

\definition
\emph{Наибольшим общим делителем (НОД)}
двух чисел называется наибольший из общих делителей этих чисел.
\emph{Наименьшим общим кратным (НОК)}
двух чисел называется наименьшее число, делящееся на каждое из них.

\begin{problems}

\item
$A = 2^3 \cdot 3^{10} \cdot 5 \cdot 7^2$,
$B = 2^5 \cdot 3 \cdot 11$.
\\
Чему равен
\quad
\sbp $\text{НОД}(A, B)$?
\quad
\sbp $\text{НОК}(A, B)$?

\item
Докажите, что
\(
    \text{НОД}(a, b) \cdot \text{НОК}(a, b) = a b
\)
для любых натуральных чисел $a$, $b$.

\item
С первого сентября четыре школьника начали посещать кинотеатр.
Первый бывал в нем каждый четвертый день, второй~--- каждый пятый, третий~---
каждый шестой и четвертый~--- каждый девятый.
Когда второй раз все школьники встретятся в кинотеатре?

\item
Найдите все целые числа $a$, для которых число $\dfrac{a^2 + 1}{a + 2}$ будет
целым.

\item
$p$ и $q$~--- различные простые числа.
Сколько делителей у числа
\\
\sbp $p q$;
\quad
\sbp $p^2 q$;
\quad
\sbp $p^2 q^2$;
\quad
\sbp $p^n q^m$?

\item
Найдите все числа, которые заканчиваются цифрами $\ldots 2013$ и уменьшаются в
целое число раз после вычеркивания этих цифр.

\item
Докажите, что произведение любых пяти последовательных чисел делится
\\
\sbp на 30;
\quad
\sbp на 120.

\item
Докажите, что для четных натуральных чисел $n$ число $n^3 + 20 n$ делится на
$48$.

\item
$p$~--- простое число.
Сколько существует натуральных чисел
\\
\sbp меньших $p$ и взаимно простых с ним;
\\
\sbp меньших $p^2$ и взаимно простых с ним?

\item
Каково наименьшее натуральное $n$ такое, что $n!$ делится на $990$?

\item
\sbp
Может ли $n!$ оканчиваться ровно на $5$ нулей?
\\
\sbp
На сколько нулей оканчивается число $(100!)$?

\item
Докажите, что простых чисел бесконечно много.

\item
Найдите все пары простых чисел, отличающихся на 17.

\item
Докажите, что найдутся 2013 подряд идущих составных чисел.

\item
Пусть $x$ и $y$~--- целые числа такие, что $3 x + 7 y$ делится на $19$.
Докажите, что $43 x + 75 y$ тоже делится на $19$.

\item
Докажите, что число $3^{54} - 3^{27} \cdot 2^{12} + 2^{24}$ является составным.

\item
$56 a = 65 b$.
Докажите, что $a + b$~--- составное число.

\item
Докажите, что число, имеющее нечетное количество делителей,~--- точный квадрат.

\item
На занятие секции по керлингу пришло 100 мальчиков.
Переодевшись в раздевалке, они заперли вещи в шкафчиках
(каждый в своем, всего 100 шкафчиков).
Во время тренировки на раздевалку был совершен налет привидений: сначала
прилетело первое и открыло все шкафчики, затем прилетело второе и изменило
положение замка в каждом втором шкафчике
(по порядку, открыто на закрыто и наоборот),
затем прилетело третье и изменило положение замка в каждом третьем шкафчике,
$\ldots$, и так далее до сотого приведения, которое изменило положение замка в
сотом шкафчике.
Затем прилетело Главное Привидение и забрало вещи из всех открытых шкафчиков.
Скольким ребятам предстоит идти после тренировки домой в спортивной форме?

\end{problems}

