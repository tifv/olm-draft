%%%%%%%%%%%%%%%%%%%%%%%
% Арифметика остатков %
%%%%%%%%%%%%%%%%%%%%%%%

\begin{problems}

\item
Докажите, что
\\\sbp
сумма любых двух натуральных чисел и сумма их остатков имеют одинаковые остатки
при делении на 3.
\\\sbp
произведение любых двух натуральных чисел и произведение их остатков имеют
одинаковые остатки при делении на 3.

\item
Найдите остатки от деления
\\
\sbp $2010 \cdot 2011 \cdot 2012 + 2013^3$ на $7$.
\quad
\sbp $9^{100}$ на $8$.

\item
Докажите, что $n^3 + 2 n$ делится на $3$ для любого $n$.

\item
Докажите, что $n^2 + 1$ не делится на $3$ ни при каком натуральном $n$.

\item
Докажите, что $n^3 - n$ делится на $24$ при любом нечетном $n$.

\item
Докажите, что
\quad
\sbp
$p^2 - 1$ делится на $24$, если $p$~--- простое число и $p > 3$.
\\
\sbp
$p^2 - q^2$ делится на $24$, если $p$ и $q$~--- простые числа, большие $3$.

\item
Натуральные числа $x$, $y$, $z$ таковы, что $x^2 + y^2 = z^2$.
Докажите, что хотя бы одно из этих чисел делится на $3$.

\item
$a$ и $b$~--- натуральные числа, причем число $a^2 + b^2$ делится на $21$.
Докажите, что оно делится и на $441$.

\item
Три простых числа, большие $3$, образуют арифметическую прогрессию.
Докажите, что разность этой прогрессии делится на $6$.

\item
Докажите, что сумма квадратов трех натуральных чисел, уменьшенная на $7$, не
делится на $8$.

\item
Найдите последнюю цифру числа
\quad
\sbp $18^{2011}$;
\quad
\sbp $2^{50}$.
\\\sbp
Найдите остаток от деления $2^{100}$ на $3$.

\item
Докажите, что $2222^{5555} + 5555^{2222}$ делится на $7$.

\item
Найдите последнюю цифру числа $7^{7^{7}}$.

\item
$p$, $p + 10$, $p + 14$~--- простые числа.
Найдите $p$.

\item
$p$, $p^2 + 8$~--- простые числа.
Докажите, что $p^3 + 2$~--- также простое.

\end{problems}

\observation
Обратите внимание на то что 
\\\textbf{(1)}
часто задачи, в которых встречаются \textbf{квадраты} натуральных чисел,
решаются перебором остатков от деления на $3$ или на $4$.
Дело в том, что и при делении на $3$ и при делении на $4$ квадраты могут давать
только остатки $0$ и $1$.
\\\textbf{(2)}
часто задачи, в которых встречаются \textbf{кубы} натуральных чисел, решаются
перебором остатков от деления на $7$ или на $9$.
Дело в том, что при делении на $7$ кубы могут давать только $0$, $1$, $6$, а
при делении на $9$ кубы могут давать только остатки $0$, $1$, $8$.

\begin{problems}

\item
Докажите, что $a^3 + b^3 + 4$ не является кубом целого числа ни при каких
натуральных $a$ и $b$.

\item
Найдите последнюю цифру числа $1^2 + 2^2 + \ldots + 99^2$.

\item
Про семь натуральных чисел известно, что сумма любых шести из них делится на
$5$.
Докажите, что каждое из чисел делится на $5$.

\item
Найдите наименьшее число, дающее следующие остатки:
$1$ при делении на $2$,
$2$ при делении на $3$,
$3$ при делении на $4$,
$4$ при делении на $5$ и
$5$ при делении на $6$.

\item
Докажите, что, если $(n - 1)! + 1$ делится на $n$, то $n$~--- простое число.

\item
Докажите, что не существует таких целых чисел $n$ и $m$, что
$m^3 + 6 m^2 + 5 m = 27 n^3 + 9 n^2 + 9 n + 1$.

\item
Может ли сумма трех последовательных квадратов целых чисел быть равной сумме
кубов двух последовательных чисел?

\end{problems}

