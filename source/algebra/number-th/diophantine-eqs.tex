%
% Диофантовы уравнения
%

\emph{%
Решите уравнения, если не предлагается иное.
Решите в целых числах, если не указано иное.}

\begingroup\def\amethod{\vspace{-2ex}\subsubsection*}

\amethod{Линейные диофантовы уравнения от двух переменных}

\begin{problems}

\item
$3n + 15 m = 7$.
\qquad
\problem
$5 x - 7 y = 3$.
\qquad
\problem
$55 x - 41 y = 444$.

\end{problems}

\amethod{Метод разложения на множители}

\begin{problems}

\item
$x^2 - y^2 = 2011$\; в $\NN$.
\qquad
\problem
$x^3 + x^2 + x + 3 = 0$.
\qquad
\problem
$x^2 + x = y^2$.

\end{problems}

\amethod{Метод остатков}

\begin{problems}

\item
Докажите, что следующие уравнения не имеют решений в $\ZZ$:
\\
\sbp $x^2 + y^2 = 2011$;
%^2\mod4 
\qquad
\sbp $a^2 - 3 b^2 = 8$;
%^2\mod3 
\qquad
\sbp $x^2 + y^2 + z^2 = 7^{2011}$.
%^2\mod8 

\item
$3^m + 7 = 2^n$.
%m=4
\qquad
\problem
$1! + 2! + \ldots + n! = m^2$\; в $\NN$.
%m=8

\end{problems}

\amethod{Метод спуска}

\begin{problems}

\item
$8 x^4 + 4 y^4 + 2 z^4 = t^4$;
\qquad
\problem
$x^2 + y^2 + z^2 = 2 x y z$.

\end{problems}

\endgroup%\def\amethod

\subsection*{Разнобой}

\begin{problems}

\item\begin{tabbing}
\sbp $3^n = x^2 + y^2$;
\qquad\=
\sbp $2 x y + 3 y^2 = 24$;
\qquad\=
\sbp $x^2 + y^2 = x + y + 2$;
\\
%
\sbp $x^6 = 56 + y^3$;
\>
\sbp $3 \cdot 2^m + 1 = n^2$;
\>
\sbp $x^2 - y^2 = 303$\; в $\NN$;
\\
%
\sbp $2^x - 1 = 5^y$;
\>
\sbp $15 x^2 - 7 y^2 = 9$;
\>
\sbp $x^3 + 21 y^2 + 5 = 0$;
\\
%
\sbp $1 + x + x^2 + x^3 = 2^y$\; в $\NN$;
\qquad
\sbp $x^2 + y^2 + z^2 + u^2 = 2 x y z u$;
\\
\sbp $p^2 - 2 q^2 = 1$\; в простых числах.
\end{tabbing}

\item
Докажите, что уравнения не имеют решений в целых числах:
\\
\sbp $19 x^3 - 84 y^2 = 1984$;
%^3 \mod 7 либо ^2 \mod 19
\qquad
\sbp $n^3 + 2 = 9 k$;
%mod{9} 
\qquad
\sbp $x_1^4 + \ldots + x_{14}^4 = 2015$.
%mod{16} 

\item
На прямой сидит блоха, которая может прыгать на $8\,\text{см}$ или на
$11\,\text{см}$ вправо или влево.
Сможет ли она сместиться после нескольких прыжков на $2011\,\text{см}$ вправо
от~начального положения?

\item
Решите в целых числах уравнение $71 x - 27 y = 2011$ и найдите решение
$(x_0, y_0)$ с наименьшим произведением $x_0 \cdot y_0$.

\item
\sbp
Докажите, что сумма квадратов десяти последовательных натуральных чисел не
является точным квадратом.
\\
\sbp
Найдите 11 последовательных натуральных чисел, сумма квадратов которых является
полным квадратом.

\end{problems}

