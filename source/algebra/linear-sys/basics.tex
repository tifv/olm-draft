%%%%%%%%%%%%%%%%%%%%%%%%%%%%%%
% Системы линейных уравнений %
%%%%%%%%%%%%%%%%%%%%%%%%%%%%%%

Будем изучать системы $m$ линейных уравнений с $n$ неизвестными.
\[\left\{\begin{aligned}
    a_{11} x_1 + a_{12} x_2 + \dots + a_{1n} x_n
&=
    b_1
\\
    a_{21} x_1 + a_{22} x_2 + \dots + a_{2n} x_n
&=
    b_2
\\
    &\cdots
\\
    a_{m1} x_1 + a_{m2} x_2 + \dots + a_{mn} x_n
&=
    b_m
\end{aligned}\right.\]

Для простоты записи удобно рассматривать сокращенную запись системы~---
матрицу, соответствующую системе:
\[\left(\begin{array}{cccc|c}
    a_{11} & a_{12} & \cdots & a_{1n} & b_1
\\
    a_{21} & a_{22} & \cdots & a_{2n} & b_2
\\
    \vdots & \vdots & \ddots & \vdots & \vdots
%    \hdotsfor{4} & \hdots
\\
    a_{m1} & a_{m2} & \cdots & a_{mn} & b_m
\end{array}\right)\]

Если каждое из уравнений системы обращается в тождество после замены
неизвестных $x_i$ числами $x_i^0$, то упорядоченный набор из $n$ чисел
$(x_1^0, x_2^0, \ldots, x_n^0)$ называется \emph{решением} системы.

Система, не имеющая ни одного решения, называется \emph{несовместной}.
Если же у системы есть решения, то она называется \emph{совместной}.
Совместная система, имеющая ровно одно решение, называется
\emph{определенной}.
Если же у совместной системы решений более одного --- она называется
\emph{неопределенной}.

\subsection*{Метод Гаусса}

Говорят, что одна линейная система $m \times n$ получена из другой линейной
системы $m \times n$ \emph{элементарным преобразованием}, если
\\\textbf{(1)}
у систем все уравнения кроме $i$-го и $j$-го совпадают, а $i$-ое и $j$-ое
отличаются лишь порядком
\\\textbf{(2)}
у систем все уравнения кроме $i$-го совпадают, а коэффициенты $i$-го имеют
вид
\(
    (a_{i1} + c a_{k1}),
    (a_{i2}+c a_{k2}),
    \ldots,
    (a_{in}+c a_{kn})
|
    (b_i+c b_k)
\),
где $c$~--- произвольное число

Говорят, что одна линейная система $m \times n$ \emph{эквивалентна} другой
линейной системе $m \times n$, если обе они либо несовместны, либо совместны и
обладают одними и теми же решениями.

\begin{problems}

\item
Две линейные системы эквивалентны, если одна получается из другой применением
конечной последовательности элементарных преобразований.

\item
Любую матрицу можно при помощи элементарных преобразований
(\emph{методом Гаусса}) привести к ступенчатому виду:\pagebreak[0]
\[
\left(\begin{array}{ccccccc|c}
    \ov{a_{11}} & \hdotsfor{5} & \ov{a_{1n}} & \ov{b_1}
\\
    & \ov{a_{2k}} & \hdotsfor{4} & \ov{a_{2n}} & \ov{b_2}
\\
    & & \ov{a_{3l}} & \hdotsfor{3} & \ov{a_{3n}} & \ov{b_3}
\\
    & & & \ddots & \ddots & \ddots & \vdots & \vdots
\\
    & & & & \ov{a_{rs}} & \cdots & \ov{a_{rn}} & \ov{b_r}
\\
    & & & & & & 0 & \ov{b_{r+1}}
\\
    & & & & & & \vdots & \vdots
\\
    & & & & & & 0 & \ov{b_{m}}
\end{array}\right)
\]
Неизвестные, с которых начинаются первые $r$ уравнений называют
\emph{главными}, а остальные неизвестные называют \emph{свободными}.

\item
Для совместности системы линейных уравнений необходимо и достаточно, чтобы
после приведения к ступенчатому виду в ней не оказалось уравнений вида
$0 = \ov{b_t}$, где $\ov{b_t} \not= 0$.
Если это условие выполнено, то свободным неизвестным можно придать произвольные
значения;
главные неизвестные при этом однозначно определяются из системы.

\item
$r = n$.
Совместная система является определенной тогда и только тогда, когда в
полученной из нее ступенчатой системе нет свободных переменных.

\item
$m = n$.
Сформулируйте и докажите теоремы~--- критерии совместности и определенности
систем линейных уравнений с числом неизвестных равным числу строк в системе.

\item
$m < n$.
Сформулируйте и докажите теоремы~--- критерии совместности и определенности
систем линейных уравнений с числом неизвестных большим числа строк в системе.

\item
Докажите, что линейная система в случае $n = m$ является совместной и
определенной тогда и только тогда, когда соответствующая ей однородная система
имеет только нулевое решение.

\item
Решите системы:
\\[1ex]
\sbp
\(\left\{\begin{aligned}
&   x - 3 y + 2 z - t
=
    3
\\
&   2 x + 4 y - 3 z + t
=
    5
\\
&   4 x - 2 y + z + t
=
    3
\\
&   3 x + y + z - 2 t
=
    10
\end{aligned}\right.\)
\qquad
\sbp
\(\left\{\begin{aligned}
&   x + 2 y + 3 z - t
=
    0
\\
&   x - y + z + 2 t
=
    4
\\
&   x + 5 y + 5 z - 4 t
=
    -4
\\
&   x + 8 y + 7 z - 7 t
=
    -8
\end{aligned}\right.\)
\\[1ex]
\sbp
\(\left\{\begin{aligned}
&   x + 2 y + 3 z
=
    2
\\
&   x - y + z
=
    0
\\
&   x + 3 y - z
=
    -2
\\
&   3 x + 4 y + 3 z
=
    0
\end{aligned}\right.\)
\qquad
\sbp
\(\left\{\begin{aligned}
&   x + 2 y + 3 z - t
=
    0
\\
&   x - y + z + 2 t
=
    4
\\
&   x + 5 y + 5 z - 4 t
=
    -4
\\
&   x + 8 y + 7 z - 7 t
=
    6
\end{aligned}\right.\)
%\vspace{1ex}

\item
Может ли система линейных уравнений с действительными коэффициентами
иметь в точности два различных решения?
Постройте по двум различным решениям системы линейных уравнений с
действительными коэффициентами бесконечную серию различных решений.

\end{problems}

\subsection*{Метод Крамера}

\emph{Определителем} матрицы $2 \times 2$ назовем следующее выражение:
\[
\begin{vmatrix}
    a_{11} & a_{12}
\\
    a_{21} & a_{22}
\end{vmatrix}
=
    a_{11} a_{22} - a_{21} a_{12}
.\]

Рассмотрим систему двух линейных уравнений с двумя неизвестными:
\[
\left\{\begin{aligned}
    a_{11} x_1 + a_{12} x_2 = b_1
\\
    a_{21} x_1 + a_{22} x_2 = b_2
\end{aligned}\right.
\]
Введем следующие обозначения
\[
    \Delta
=
\begin{vmatrix}
    a_{11} & a_{12}
\\
    a_{21} & a_{22}
\end{vmatrix}
\qquad
    \Delta_{x_1}
=
\begin{vmatrix}
    b_1 & a_{12}
\\
    b_2 & a_{22}
\end{vmatrix}
\qquad
    \Delta_{x_2}
=
\begin{vmatrix}
    a_{11} & b_1
\\
    a_{21} & b_2
\end{vmatrix}
\]

\begin{problems}

\item
Пусть дана система двух линейных уравнений с двумя неизвестными, и пусть каждая
переменная входит с ненулевым коэффициентом хотя бы в одно из уравнений системы.
Тогда:\par\smallskip
Система уравнений определена и совместна,
если ее главный определитель $\Delta \not= 0$, при этом ее решение находится
по формулам
$x_1 = \Delta_{x_1} / \Delta$,
$x_2 = \Delta_{x_2} / \Delta$.
Если же главный определитель $\Delta = 0$, то в случае
$\Delta_{x_1} = \Delta_{x_2} = 0$ система совместна, но не определена.
Во всех остальных случаях система несовместна.

\item
Решите системы уравнений.
Для каждой из них выясните, при каких значениях параметров система не имеет
решений, а при каких имеет бесконечно много решений.\\[1ex]
\sbp
\(\left\{\begin{aligned}
&   a x + y = a^2
\\
&   x + a y = 1
\end{aligned}\right.\)
\qquad
\sbp
\(\left\{\begin{aligned}
&   a x + a y = a^2
\\
&   x + a y = 2
\end{aligned}\right.\)
\qquad
\sbp
\(\left\{\begin{aligned}
&   (a + 1) x + 8 y = 4 a
\\
&   a x + (a + 3) y = 3 a - 1
\end{aligned}\right.\)
\\[1ex]
\sbp
\(\left\{\begin{aligned}
&   a^2 x + (2 - a) y = 4 + a^2
\\
&   a x + (2 a - 1) y = a^5 - 2
\end{aligned}\right.\)
\qquad
%\sbp
%\(\left\{\begin{aligned}
%&   a x + y = a^3
%\\
%&   x + a y = 1
%\end{aligned}\right.\)
%\qquad
%\sbp
%\(\left\{\begin{aligned}
%&   a x - a y = a b
%\\
%&   2 a x - y = a
%\end{aligned}\right.\)
%\\[1ex]
\sbp
\(\left\{\begin{aligned}
&   a x + b y = a
\\
&   b x + a y = b
\end{aligned}\right.\)
\qquad
\sbp
\(\left\{\begin{aligned}
&   |a| x - y = 1
\\
&   x + |a| y = a
\end{aligned}\right.\)
\vspace{1ex}

\end{problems}

\subsection*{Взаимное расположение прямых}

\begin{problems}

\item \label{lines in coordinates}
Две прямые заданы своими уравнениями:
\[\left\{\begin{aligned}
    a_1 x + b_1 y + c_1 = 0\\
    a_2 x + b_2 y + c_2 = 0
\end{aligned}\right.\]
Сформулируйте и докажите условия, накладываемые на коэффициенты прямых, если
\quad
\sbp прямые пересекаются в одной точке;
\\
\sbp прямые параллельны и не совпадают;
\quad
\sbp прямые совпадают.

\item
Треугольник задан координатами своих вершин
$(x_1; y_1)$, $(x_2; y_2)$, $(x_3; y_3)$.
\\
\sbp Составьте уравнения прямых, содержащих медианы треугольника.
\\
\sbp Докажите, что в треугольнике медианы пересекаются в одной точке.

\item
Обобщите задачу \ref{lines in coordinates} на случай трех плоскостей в
пространстве.
Докажите соответствующие утверждения.

\item
Назовем медианами тетраэдра прямые, проходящие через вершины и центры тяжести
противоположных граней.
Верно ли, что медианы тетраэдра для любого тетраэдра пересекаются в одной
точке?

\end{problems}

\subsection*{Разные задачи}

\begin{problems}

\item
Найти квадратный трехчлен $f(x)$, зная, что
$f(1) = -1$, $f(-1) = 9$, $f(2) = -3$.

\item
За круглым столом сидят $4$ гнома.
Перед каждым стоит кружка с молоком.
Один из гномов переливает $1/4$ своего молока соседу справа.
Затем сосед справа делает то же самое.
Затем то же самое делает следующий сосед справа и, наконец, четвертый гном
$1/4$ оказавшегося у него молока переливает первому.
Во всех кружках вместе молока $2$ литра.
Сколько молока первоначально было в кружках, если
\\
\sbp в конце у всех гномов молока оказалось поровну?
\\
\sbp в конце у всех гномов оказалось молока столько, сколько было в начале?

\item Имеется система линейных уравнений
\[\left\{\begin{aligned}
    \ast x + \ast y + \ast z = 0
\\
    \ast x + \ast y + \ast z = 0
\\
    \ast x + \ast y + \ast z = 0
\end{aligned}\right.\]
Два человека вписывают по очереди вместо звездочек действительные числа.
Докажите, что начинающий всегда может добиться того, чтобы система имела
ненулевое решение.

\item
Функция $f(x)$ при каждом действительном значении $x$ удовлетворяет равенству
\[
    f(x)
    +
    (x + 1/2) \cdot f(1 - x)
=
    1
.\]
\sbp
Найдите $f(0)$ и $f(1)$.
\quad
\sbp
Найдите все такие функции.

\item
Решите системы уравнений:
\\[1ex]
\sbp
\(\left\{\begin{aligned}
&   x_1 + x_2 + x_3 = 0
\\
&   x_2 + x_3 + x_4 = 0
\\[-0.5ex]
&   \cdots
\\[-0.5ex]
&   x_{99} + x_{100} + x_1 = 0
\\
&   x_{100} + x_1 + x_2 = 0
\end{aligned}\right.\)
\qquad
\sbp
\(\left\{\begin{aligned}
&   x + y + z = a
\\
&   x + y + t = b
\\
&   x + z + t = c
\\
&   y + z + t = d
\end{aligned}\right.\)
\\[1ex]
\sbp
\(\left\{\begin{aligned}
&   x_1 + x_2 + x_3 + x_4 = 2 a_1
\\
&   x_1 + x_2 - x_3 - x_4 = 2 a_2
\\
&   x_1 - x_2 + x_3 - x_4 = 2 a_3
\\
&   x_1 - x_2 - x_3 + x_4 = 2 a_4
\end{aligned}\right.\)
\qquad
\sbp
\(\left\{\begin{aligned}
&   x_1 + 2 x_2 + 3 x_3 + \ldots + n x_n = a_1
\\
&   n x_1 + x_2 + 2 x_3 + \ldots + (n - 1) x_n = a_2
\\[-0.5ex]
&   \cdots
\\[-0.5ex]
&   2 x_1 + 3 x_2 + 4 x_3 + \ldots + x_n = a_n
\end{aligned}\right.\)

\end{problems}

