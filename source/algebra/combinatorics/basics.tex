%%%%%%%%%%%%%%%%%%%%%%%%%%%%%%%%%%
% Перечислительная комбинаторика %
%%%%%%%%%%%%%%%%%%%%%%%%%%%%%%%%%%

% Владимир Шарич, whenever

\setcounter{footnote}{0}

\begin{problems}

\item
Сколько семизначных чисел, в десятичной записи которых есть единица?

\item
Сколько всего подмножеств множества $M$ из $n$ элементов, включая пустое и само
$M$?

\item
В некоторой школе каждый школьник знаком с $32$ школьницами, а каждая
школьница~--- с $29$ школьниками.
Кого в школе больше: школьников или школьниц и во сколько раз?

\item
Семнадцать девушек водят хоровод.
Сколькими различными способами они могут встать в круг?
Как изменится это число, если известно, что среди этих девушек ровно пять
рыжеволосых и требуется, чтобы никакие две из них не стояли рядом?

\item
Сколько существует различных ожерелий из 17 попарно разноцветных бусин?

\item
Сколькими способами можно поставить на шахматную доску
\\
\sbp белого и черного
\\
\sbp двух белых королей так, чтобы они не били друг друга?

\item
Сколькими способами можно разбить класс из 24 человек
\\
\sbp на 2 необязательно равные группы?
\\
\sbp на 4 волейбольные команды?

\item
Сколькими способами можно выбрать двух мужчин и двух женщин на четыре различные
должности, если имеется десять кандидатов на эти должности, из которых
шестеро~--- мужчины и четверо~--- женщины?

\item
Сколькими способами можно представить число $n$ в виде суммы $k$ натуральных
слагаемых (суммы, отличающиеся порядком слагаемых, считаются различными)?

\item
При раскрытии скобок в выражении
$\left(x_1 + x_2 + \ldots + x_n\right)^m$
получается сумма выражений вида
$x_1^{k_1} x_2^{k_2} \ldots x_n^{k_n}$,
в которых
$k_1 \geq 0$, $k_2 \geq 0$, \ldots, $k_n \geq 0$,
$k_1 + k_2 + \ldots + k_n = m$.
\\
\sbp Сколько раз встречается каждое из таких слагаемых?
\\
\sbp Сколько всего различных слагаемых в этой сумме?

\item
Никакие три диагонали выпуклого $n$-угольника не пересекаются в одной точке,
отличной от вершины.
Сколько всего точек пересечения (отличных от вершин) получится, если провести
все диагонали?

\item
Дано $n$ точек,
$m_1 \geq 3$ из которых лежат на прямой $l_1$,
$m_2 \geq 3$ из которых лежат на прямой $l_2$,
\ldots,
$m_k \geq 3$ из которых лежат на прямой $l_k$.
Прямые $l_1$, $l_2$, \ldots, $l_k$ различны, и для любых трех точек верно, что
если они лежат на одной прямой, то все три лежат на одной из прямых
$l_1$, $l_2$, \ldots, $l_k$.
Сколько всего прямых, каждая из которых содержит хотя бы две из этих точек?

\end{problems}

\subsection*{Биномиальные коэффициенты%
\footnote{По умолчанию суммирование ведется по всем допустимым значениям
переменной
(т.\,е. в ${\binom{a}{b}}$ числа $a \geq b \geq 0$).}}

\newcommand{\ds}{\displaystyle}

\begin{problems}

\item
\emph{Хромым королём} назовем <<шахматную>> фигуру, которая может ходить только
на одну клетку и только вправо и вниз.
Допустим, хромой король стоит в левом верхнем углу бесконечной вправо и вниз
<<шахматной>> доски, вертикали и горизонтали которой занумерованы,
соответственно, справа налево и сверху вниз неотрицательными целыми числами
(начиная с нуля).
Сколькими способами хромой король может добраться с поля $(0, 0)$ на поле
$(p, q)$?

\item\emph{Джентльменский набор.}
Докажите и запомните навсегда, что
\\[1ex]
\sbp
\(\ds
    \sum_{k = 0}^n
        \binom{n}{k}
=
    2^n
\);
\qquad
\sbp
\(\ds
    \sum_{k = 0}^n
        (-1)^{k} \binom{n}{k}
=
    0
\);
\qquad
\sbp
\(\ds
    k \binom{n}{k}
=
    n \binom{n-1}{k-1}
\);
\\[0.5ex]
\sbp
\(\ds
    (n - k) \binom{n}{k}
=
    n \binom{n-1}{k}
\);
\qquad
\sbp
\(\ds
    \binom{n}{k}
=
    (-1)^k \binom{k - n - 1}{k}
\);
\\[1ex]
\sbp\emph{Свертка Вандермонда:}
\;
\(\ds
    \sum_{k}
        \binom{p}{k}
        \binom{q}{n - k}
=
    \binom{p + q}{n}
\).

\item
\sbp
\(\ds
    \binom{n}{m} \binom{m}{k}
=
    \binom{n}{k} \binom{n - k}{m - k}
=
    \binom{n}{m - k}
    \binom{n - m + k}{k}
\);
\\[1ex]
\sbp
\(\ds
    \sum_{k}
        \binom{n - k}{k}
=
    F_n
\).

\item
\sbp
\(\ds
    \sum_{k \leq m}
        \binom{n}{k}
        \left(
            \dfrac{n}{2} - k
        \right)
=
    \frac{m + 1}{2}
    \binom{n}{m + 1}
\)
для $n\geq m$
\\[0.5ex]
\sbp
\(\ds
    \sum_{k}
    \binom{n - m}{k}
    x^k y^{n - k}
=
    \sum_{k}
        \binom{m}{k}
        (-x)^k (x + y)^{n - k}
\);
\\[0.5ex]
\sbp
\(\ds
    \sum_{k}
        \binom{n - m}{k}
        (-1)^k
=
    \binom{m}{n}
\).

\end{problems}

\subsection*{Рекурренты}

\begin{problems}

\item
\sbp
\(\ds
    \sum_{n \geq 0}
        F_n x^n
=
    \frac{1}{1 - x - x^2}
\), где $F_n$~--- числа Фиббоначи, $F_0 = F_1 = 1$.
\\[0.5ex]
\sbp
\(\ds
    \sum_{k \leq n}
        \binom{n + k}{k} 2^{-k}
=
    2^n
\).

\end{problems}

\subsection*{Числа Каталана}

\definition
\emph{Число Каталана}~--- элемент последовательности $C_n$, $n \geq 0$, заданной
первым членом $C_0 = 1$ и рекуррентным соотношением
\(
    C_n
=
    C_0 C_{n - 1}
    +
    C_1 C_{n - 2}
    +
    \ldots
    +
    C_{n - 1} C_0
\).

\begin{problems}

\item
Вычислите $C_1$, $C_2$, $C_3$, $C_4$.

\item
Докажите, что:
\quad
\sbp
%\footnote{%
%Это можно \emph{вывести}, но для этого надо уметь работать с формальными
%степенными рядами.}%
\(\ds
    C_n
=
    \frac{1}{n + 1}
    \binom{2n}{n}
\).
\\[0.5ex]
\sbp
$C_n$ равно количеству способов разбить выпуклый $(n + 2)$--угольник на
треугольники непересекающимися диагоналями.
\\
\sbp
$C_n$ равно количеству способов расставить в ряд $n$ открывающихся и $n$
закрывающихся скобок так, чтобы запись была корректна
(то есть, среди любого количества первых элементов ряда открывающихся скобок не
меньше, чем закрывающихся).
Например, если $n = 3$, то таких способов пять:
\mbox{\tt((()))}, \mbox{\tt(())()}, \mbox{\tt()(())}, \mbox{\tt(()())},
\mbox{\tt()()()}.
Следовательно, $C_3 = 5$.
\\
\sbp
$C_n$ равно количеству бинарных деревьев с $n + 1$ листьями.
Бинарным называется дерево с выделенной вершиной (корнем) степени 2, все
остальные вершины которого имеют степень 1 или 3.
\\
\sbp
$C_n$ равно количеству путей из точки $(0, 0)$ в точку $(n, n)$ по линиям
клетчатой бумаги, идущих вверх и вправо и не поднимающихся выше прямой $y = x$.

\let\ds\undefined

\end{problems}

\subsection*{Формула включения--исключения и принцип Дирихле}

\begin{problems}

\item
Сколько существует целых чисел от 1 до $10^{30}$, которые не являются ни полным квадратом, ни полным кубом, ни пятой степенью?

\item\emph{Из творчества Льюиса Кэрролла.}
В ожесточенном бою более 70 из 100 пиратов потеряли один глаз, более 75~---
одно ухо, более 80~--- одну руку и более 85~--- одну ногу.
Каково наименьшее количество пиратов, потерявших одновременно глаз, ухо, руку и
ногу?

\item
В квадрате площадью 6 расположены 3 многоугольника площадью 3 каждый.
Докажите, что среди них найдутся два многоугольника, площадь общей части
которых $\geq 1$.

\item
В классе 25 учеников.
Сколькими способами они могут пересесть так, чтобы ни один ученик не сел на
свое место?

\item
В ряд выписано двенадцать натуральных чисел:
$a_1$, $a_2$, \ldots, $a_{12}$.
Докажите, что сумма одного или нескольких рядом стоящих чисел делится на 12.%
%подсказочка
%\footnote{%
%Из любых 13 натуральных чисел можно выбрать два, разность которых
%делится на 12.}

\item
В квадрате со стороной $1$ отметили $51$ точку.
Докажите, что три из них можно покрыть кругом радиуса $\dfrac{1}{7}$.

\item
В круге радиуса $19 / 2$ размещены $401$ точка.
Докажите, что среди них можно выбрать две точки, расстояние между которыми не
превосходит 1.

\item
\emph{Теорема Блихфельда.}
Докажите, что плоское множество площади больше $n$ можно сдвинуть
(не разворачивая) так, чтобы оно покрывало хотя бы $n + 1$ узел целочисленной
решетки.

\end{problems}

